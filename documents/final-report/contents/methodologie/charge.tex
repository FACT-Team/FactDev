\section{L'évaluation des charges de travail}
\begin{tabular}{|l|p{2.6cm}|p{2.6cm}|p{2.3cm}|p{2.3cm}|c|}
	\hline
	\textbf{Bilan} & & \textbf{Prévu} & \textbf{Réel} & \textbf{Diff}& \textbf{Ecart}\\
	\hline
	& Date de début &27/12/2014 &27/12/2014&&N/A\\
	\hline
	& Date de fin &20/04/2015 &20/04/2015&&N/A\\
	\hline
	& Charges groupe &432 &480&\texttt{+48}&\texttt{+11\%}\\
	\hline
	& Charges encadrant &12 &9&\texttt{-3}&\texttt{-25\%}\\
	\hline
\end{tabular}

\subsection{Explication des écarts}
\subsubsection{Pour le groupe}
Durant les quatres premiers \Sprints{} nous avons respecté le temps que nous avions prévu de consacrer au projet à savoir 24 heures par semaine par
développeur. À ceci s'ajoute le temps consacré pour les réunions qui est resté inchangé durant la totalité du projet. Cependant, sur les deux derniers
\Sprints, nous
avons dû ajouter des \Stories{} à notre projet ce qui a augmenté la quantité de travail à fournir.

\subsubsection{Pour l'encadrant}
Nous avions prévu de voir l'encadrant de façon hebdomadaire afin qu'il puisse nous suivre et nous « coacher ». Cependant, une fois la première release
terminée, étant donné notre démarche qualité et notre avancement dans notre projet, nous avons décidé de nous rencontrer toutes les deux semaines, lors
des revues de sprints.

