\section{L'intégration continue}\label{integration-continue}
\subsection{L'utilisation d'un logiciel de versionnement : Git}
\key{Git} est un outil puissant permettant de conserver les différentes versions de notre logiciel, de connaître les fonctions auxquelles sont affectées les membres de l'équipe et, en cas de difficultés, de revenir dans un état antérieur. 

\key{Git} permet également, via le système des branches exposé plus haut, de travailler sur des versions séparées du code. Ainsi, il est possible de développer chacun de son côté sans créer de conflits avec le code des autres membres de l'équipe. 
% La fusion automatique
% Github pour la communication, Issues, TODO/Done/current

\subsection{Le Branching Workflow}
Afin de pouvoir facilement travailler en parallèle, et pour optimiser les algorithmes de fusion de Github, nous avons utilisé un système de « Git
Branching ». Celui-ci, comme le montre la figure \ref{fig:branching}, contient une branche par fonctionnalités, chaque fonctionnalité est en suite
fusionnée dans son sprint après une revue de code présentée dans la section \ref{revue}.
\begin{figure}[H]
	\centering
	\includegraphics*[width=13cm]{BranchingWorkflow.eps}
	\caption{Principe du « Git Branching Workflow »}
	\label{fig:branching}
\end{figure}

\subsection{L'intégration continue avec \Travis}
Afin d'avoir un logiciel de la meilleure qualité possible, nous avons utilisé \Travis. Celui-ci va effectuer un certain nombre d'actions à chaque
envoie de code sur \Github. Une fois ces action exécutés, il est ensuite capable de nous donner un verdict sur le code à cet instant T :
\textit{passed} ou \textit{failed}.

Ainsi, Travis va effectuer les actions suivantes, comme le montre la figure \ref{fig:travis}:
\begin{description}
	\item[Compilation] Compilation de l'intégralité du projet.
	\item[Tests] Exécution de l'ensemble des tests.
	\item[Coverage] Calcul de la couverture de code des tests précédemment exécutés. 
	\item[Doxygen] Génération de la documentation technique, au format HTML, PDF, ainsi que du manuel utilisateur.
\end{description}

Dans une le cas où une de ces actions échoue, alors le build est en échec et le développeur doit corriger son code.
\begin{figure}[H]
	\centering
	\includegraphics*[width=18cm]{travis.eps}
	\caption{Fonctionnement de \Travis}
	\label{fig:travis}
\end{figure}

