\chapter*{Conclusion}
Nous tenons tout d'abord à remercier toutes les parties prenantes au projet: notre encadrant M. \bsc{Migeon} qui nous apporter son expérience dans le développement du logiciel et une vision neutre sur le travail fourni ; M. \bsc{Cherbonneau} pour le suivi de l'ensemble des élèves dans le cadre de cette UE. 

Le projet \FactDev{} nous a permis de nous placer en situation professionnelle: nous avons dû apprendre à nous adapter aux exigences de notre client (Antoine) ou de notre encadrant (M. \bsc{Migeon}) ainsi que mettre en place des moyens de communication et d'organisation pour parvenir à nos objectifs. 

Il représente une expérience valorisante que l'on peut mettre en avant dans nos CV et lors de futurs entretiens. Sur le plan technique, nous pouvons citer le C++ (et le framework Qt), le \LaTeX, les outils \Travis{}, \Github{} et \Coveralls. Sur le plan organisationnel avec l'application stricte de la méthode \Scrum{} et le respect de la méthodologie mise en place pour assurer de la qualité du logiciel. 

Bien que l'ensemble des fonctionnalités n'ont pu être implémentés et une difficulté à assurer nos \Melees{} quotidiennes le résultat qui en ressort est très positif. Le logiciel \FactDev{} est fonctionnel, distribué sous licence GPL et est déjà utilisé par Antoine. 