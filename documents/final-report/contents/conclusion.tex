\chapter*{Conclusion}
Nous tenons avant tout à remercier toutes les parties prenantes de ce projet qui nous ont permis de le mener à bien que se soit l'équipe pédagogique de l'université, le client ou l'équipe de développement. 

Le projet FactDev nous a en effet permis de pouvoir mener un projet comme dans le cadre d'une entreprise dans le monde professionnel. Malgré les difficultés que nous avons pu rencontrer comme la difficulté de communiquer, d'exprimer nos points de vues, nos problèmes, des fonctionnalités que nous n'avons pu implanter, de nombreux points positifs en ressortent. 

Au delà de l'expérience professionnelle que nous avons acquise grâce à la pédagogie mise en place et du plus dans nos CV, plusieurs points bénéfiques en ressortent.
Tout d'abord le logiciel sera sous licence GPL, il sera utilisable et utilisé par son client et éventuellement d'autres personnes dans le futur. Au niveau technique, une multitude de compétences ont été acquises, les principales sont le langage C++ ainsi que les méthodes agiles mais d'autres moins principales mais tout aussi nécessaires ont été vues comme la gestion du projet avec GitHub ou encore les tests avec Travis et Coveralls.

C'est pourquoi cette pédagogie innovante qui nous permet d'acquérir une certaine autonomie est nécessaire dans un monde qui évolue tout les jours notamment celui du numérique avec ses innovations.