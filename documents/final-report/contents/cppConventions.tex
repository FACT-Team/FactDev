\chapter{Conventions de programmation en C++}
Voici les conventions d'écriture que nous avons fixé, il faudra les
respecter pour que nous ayons un code propre et homogène, de plus elles
ont été fixées pour que ce soit le plus simple pour nous (lecture rapide,
propreté etc\ldots{}).

\begin{attention}
Elles peuvent encore évoluer
\end{attention}

\section{English, of course !}\label{english-of-course}

Tout le code, et les \texttt{commits}, doivent être rédigés en Anglais. Soit, les
noms de classes, de méthodes, de fichiers, de variables, d'attributs, et
même les commentaires ;-) C'est pas compliqué, mais au moins, on se met
tous d'accord, et puis voilà :)

\section{Le nommage}\label{le-nommage}

\subsection{Les attributs}\label{les-attributs}

Les attributs doivent toujours commencer avec une minuscule, pour
séparer les mots, on les sépare avec une majuscule : utilisation de la
Camel Case. Les noms de variables claires et explicite, quitte à ce
qu'il soit un peu long, on a l'auto complétion que diable! Donc les
variables d'une lettre, à bannir! (à part le i dans le cas d'un for,
s'il n'est pas réutilisé après le for)

Les attributs seront préfixés par \texttt{\_} afin de pouvoir les reconnaître
facilement.

\begin{lstlisting}[language=C++,numbers=none]
int _superField; 
bool _youLostTheGame;
\end{lstlisting}

\subsection{Les noms de constantes}\label{les-noms-de-constantes}

Les constantes doivent être tout en majuscule, les différents mots de la
constante sont séparés par des underscore (\texttt{\_}). Même remarque que pour
les attributs, choisissez des noms de constantes clairs, compréhensibles
par tous, pas seulement par ceux qui sont dans votre tête!

\begin{lstlisting}[language=C++]
const int MY_BEAUTIFUL_CONSTANT; 
const bool YOU_LOST_THE_GAME_AGAIN; 
const QString CONST;
\end{lstlisting}

\begin{exemple}
	On évite d'utiliser les \texttt{\#define} afin de garder un typage fort
\end{exemple}

\subsection{Les noms de méthodes}\label{les-noms-de-muxe9thodes}

Les méthodes doivent commencer par une minuscule, et séparer les
différents mots par une majuscule(Camel case).

Les fonctions ne retournant rien (procédures) doivent toujours être à
l'infinitif.

\`A l'opposé les fonctions 
retournant quelques choses doivent être au
participe passé. Il faut décomposer au maximum, n'hésitez pas à faire
une méthode \texttt{private} si besoin est, c'est toujours plus clair
d'avoir une fonction, dictant explicitement ce qu'elle fait par son nom
que 10 lignes de code bizarroïdes avec 2-3 lignes de commentaire! Et donc, les noms de fonctions sont essentiels!!

\begin{lstlisting}[language=C++]
void display(QString textToDisplay) { 
    qDebug() << texteAAffiche; 
}

bool test(int first, int second) { 
	return (first + second);
} 
\end{lstlisting}

\paragraph{Les accesseurs}\label{les-accesseurs}

Les getters et les setters doivent être respectivement préfixés par
\texttt{get} et \texttt{set} suivi du nom \emph{exact} de l'attribut,
même si le get ne respecte pas la convention de Qt.

\begin{lstlisting}[language=C++]
int getValue(); 
void setValue(int);
\end{lstlisting}

\subsection{Les noms de classes ou d'interface}\label{les-noms-de-classes-ou-dinterface}

Les noms de classe ou d'interface doivent tous commencer par une
majuscule, les différents mots sont séparés par une majuscule,
choisissez des noms de classes claires! (oui, je me répète, mais c'est
ce qui fait toute la compréhension facile, ou non, d'un programme les
noms de variables, classe, paramètre, méthodes etc\ldots{} )

\section{L'indentation}\label{lindentation}

La règle est simple, on ouvre une accolade, la ligne suivante sera
décalée vers la droite(une tab = 4 caractères), on ferme une accolade, on
décale l'accolade vers la gauche et tout ce qui suit.

Également, si une ligne est trop longue, on va à la ligne, et décalons
d'une ligne vers la droite, une fois l'instruction finie, on redécale
vers la gauche.

Dans le cas d'un switch, le break doit s'aligner avec le case 42: tout
ce qui est entre case et break sera indenté.

Merci de mettre un espace avant chaque accolade, oui je sais je suis
psychorigide, mais c'est moche sinon.

\begin{lstlisting}[language=C++]
class MaSuperClasse { 
public: 
	MaSuperClasse(int test, QString, machin, double _chose);
	int method();

private: 
	int _test;
	QString _machin;
	double _chose;
};

MaSuperClasse::MaSuperClasse(int test, QString machin, double chose) {
	_test = test;
	_machin = machin; 
	_chose = chose; 
}

MaSuperClasse::method() {
	qDebug() << "Hello World";
	switch(yatta) {
		case 42:
		   // ^^
		   break;
		case 1337:
		   // ...
		   break;
		default: 
		   //
	}
}
\end{lstlisting}

\section{Les accolades}\label{les-accolades}

Les accolades ouvrantes sont positionnées à la fin de la ligne demandant
une accolade (\texttt{switch}, \texttt{if}, \texttt{class}, \texttt{else}, \texttt{else if}, \ldots{})

Les accolades fermantes sont positionnées une ligne après la dernière
instruction. (avec une désindentation) Les \texttt{else} et \texttt{else if} se mettent sur la 
même ligne que l'accolade fermante.


\begin{lstlisting}[language=C++]
int superMethod(void) {     
	if(true) {
		// bla bla     
	} else if(false) {
		// bla bla      
	} else { 
		//instruction 
	}
}
\end{lstlisting}

\section{Les types}\label{les-types}

Pour les types, au maximum, il vaut mieux privilégier les classes de Qt
plutôt que les Types C++, autrement dit, on va utiliser les types
suivants(c'est facile, ça commence pas un Q :) ) :

\begin{itemize}
\itemsep1pt\parskip0pt\parsep0pt
\item
  Primitives : \texttt{int}, \texttt{double}, \texttt{char},
  \texttt{unsigned}, \texttt{bool}, \ldots{}
\item
  \texttt{QString}, \texttt{QVariant}, \texttt{QNumber}, \texttt{QDate},
  \texttt{QTime}, \texttt{QDateTime}
\item
  Collections : \texttt{QList\textless{}Type\textgreater{}},
  \texttt{QSet\textless{}Type\textgreater{}},
  \texttt{QStack\textless{}Type\textgreater{}},
  \texttt{QQueue\textless{}Type\textgreater{}},
  \texttt{QLinkedList\textless{}Type\textgreater{}},
  \texttt{QVector\textless{}Type\textgreater{}},
  \texttt{QMap\textless{}Type1, Type2\textgreater{}} \ldots{}
\end{itemize}

Je vous ferais p'têtre un wiki sur les types cool en Qt\ldots{} Mais
sinon la doc est vraiment très bien faite ! Au pire, c'est assez intuitif
« je veux une pile\ldots{} Ça commence par Q. Comment on dit pile ?
Stack ? Ah ben \texttt{QStack}. »

\section{Les bonnes pratiques}\label{les-bonnes-pratiques}

\subsection{Le mot clé \texttt{const}}\label{le-mot-cluxe9-const}

Dès que vous pouvez\ldots{} hop un \texttt{const}.

Autrement dit : si vous ne modifiez jamais un paramètre, un attribut, une
variable où que sait-je, on met un const. Ça permet de n'avoir aucune
ambiguïté, c'est clair, et quelqu'un qui utilise la méthode sait que le
paramètre ne serait pas modifié.

\subsection{Le mot clé \texttt{void}}\label{le-mot-cluxe9-void} 
Une méthode ne contenant aucun paramètre \emph{doit} contenir void,
c'est comme ça.

\begin{lstlisting}[language=C++,numbers=none]
void aSuperMethod(void);
\end{lstlisting}

\subsection{Longueur du code}\label{longueur-du-code}

Une ligne ne doit pas excéder 100 caractères, une méthode ne doit pas
excéder 60 lignes, un fichier doit être assez court\ldots{} Mais ça on
verra sur le moment ;-)

\section{Conventions des composants
graphiques}\label{conventions-des-composants-graphiques}
\begin{tabular}{c|c}
Element & Préfix\\
\hline
Combo & cb\\
Menus & mn\\
LineEdit & le\\
Buttons & btn\\
Table & tbl\\
Label & lb\\
TextEdit & te\\
Tree & tr\\
Dock & dck\\
Checkbox & chk\\
WIdget & wdg\\
\end{tabular}

\section{Conventions de documentation}\label{conventions-de-documentation}

\subsubsection{Pour une classe}\label{pour-une-classe}


\begin{lstlisting}[language=C++,numbers=none]
/**
 * @author Nom de(s) la personne ayant programmé la classe sous la forme: Prénom Nom
 * @brief The MaClass class (généré automatiquement) Role de la classe 
 */
\end{lstlisting}

\subsubsection{Pour un attribut ou une méthode}
\begin{lstlisting}[language=C++,numbers=none]
/**
 * @brief MaClass\dotsmaMethode Ce que fait la méthode
 * @see [optionnelle] Si cela fait référence à une autre méthode/classe/objet alors on écrit le nom de cette
 * méthode/classe/objet
 * @param parametre1 Brève description du paramètre attendu et de ce qu'il représente 
 * @param parametre2 \ldots
 * @param parametreN \ldots
 * @return ce qui est retourné
 */
\end{lstlisting}

\subsubsection{Remarques}
On peut utiliser les balises HTML pour la documentation. Par exemple, dans @brief si on a une méthode addCustomer(int id), sa description pourrait
être: 
\begin{lstlisting}[language=C++,numbers=none]
/**
 * @brief Customer\dotsaddCustomer Ajoute un nouveau client 
 *  possédant l'identifiant <i>id</i>``. 
 */
\end{lstlisting}

On utilisera alors comme convention: - Pour un paramètre << pParam >> passé en paramètre: 
\begin{lstlisting}[language=C++,numbers=none]
/**
 * ...<i>pParam</i>
 */
\end{lstlisting}

- Pour le nom d'une classe << MaClass >> 
\begin{lstlisting}[language=C++,numbers=none]
/**
 * ...<b>MaClass</b>
 */
\end{lstlisting}

\begin{attention}
Éviter les accents dans la documentation (Par exemple, ça sera @author Cedric Rohaut et non Cédric Rohaut)
\end{attention}
