\section{Les outils}
Afin de respecter les attentes du client, et de développer correctement le logiciel présenté section \ref{logiciel}, nous avons utilisés un
certain nombres d'outils, dans divers domaines.

\subsection{Outils de développement}
Afin de développer à proprement parler le logiciel, nous avons utilisés plusieurs technologies. 

\begin{wrapfigure}{r}{0.4\textwidth}
\begin{center}
\includegraphics[width=0.38\textwidth]{../beamer/logos/qt.png}
\end{center}
\caption{Le développement -- Qt}
\end{wrapfigure}
Proposé par le client, il a été choisi de développer le logiciel en C++, en utilisant le framework Qt. 
En effet, ce langage permet d'avoir un logiciel qui soit rapide, et peu gourmand en mémoire. Cependant, il était nécessaire d'utiliser un
framework de développement, et ce pour plusieurs raisons : 
\begin{itemize}
	\item Développer rapidement une interface claire, uniforme
	\item Avoir un logiciel Multi-Plateforme, executable sous Windows, Linux et Mac OS
	\item Faciliter les lectures et écritures à une base de données 
\end{itemize}
En plus de ces avantages certains, Qt améliore le langage C++ afin de garder la même rapidité d'exécution tout en simplifiant l'écriture de
certains concepts\footnote{Simplification de la gestion de la mémoire, Ajout du \textit{foreach}, redéfinition de tous les types de bases, …}.

\newpage
\begin{wrapfigure}{l}{0.3\textwidth}
\begin{center}
	\Huge \LaTeX
\end{center}
\caption{La mise en formex -- \LaTeX}
\end{wrapfigure}
Un des besoins du client, était la génération des factures et des devis au format PDF. Avant le développement du logiciel, ce besoin était
fait << à la main >>, en \LaTeX{}, avec un \textit{template} rédigé. Afin de regarder la même mise en forme des devis et des factures, nous
devions générer du \LaTeX{} en réutilisant ce template.\\ Une fois ce fichier \texttt{.tex} généré, nous faisons appel à un compilateur \LaTeX{} afin d'en sortir un fichier PDF.

\begin{wrapfigure}{r}{0.4\textwidth}
\begin{center}
\includegraphics[width=0.18\textwidth]{../beamer/logos/sqlite.png}~
\includegraphics[width=0.18\textwidth]{../beamer/logos/mysql.png}
\end{center}
\caption{Bases de données}
\end{wrapfigure}
Le logiciel utilise une base de données pour sauvegarder les différents clients, devis, factures etc… Le besoin initial était de pouvoir
sauvegarder ça sur un ordinateur : le système de SQLite permet cela très simplement, tout est sauvegardé dans un fichier binaire.\\ Une fois
avancé dans le projet, une autre solution est apparu : posséder un serveur de base de données afin d'utiliser le logiciel avec plusieurs
postes clients. Ce besoin a été couvert à l'aide de MySQL, ainsi le logiciel permet de choisir l'une ou l'autre des manières de procéder.


\subsection{Versionnement}


\subsection{Qualité du code}


\subsection{Organisation}
