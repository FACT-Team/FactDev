\chapter*{Introduction}

\setlength{\parindent}{1cm}
FactDev est un logiciel de devis et de facturation réalisé dans le cadre de l’UE Projet. 

Ce projet s’est fait en réponse à un problème de l’un des membres du groupe : Antoine De Roquemaurel. En effet Antoine, développeur Freelance, rédigeait pour ces clients les factures et devis « à la main ». La tâche était répétitive et des risques d’erreurs humaine important : 
\begin{itemize}
	\item erreur dans le calcul des montants
	\item perte de facture
	\item plusieurs factures différentes pour un même projet et donc risque de ne pas faire le bon travail demandé par le client
\end{itemize}

Face à ces difficultés, Antoine a soumis le projet devant répondre aux spécifications suivantes :
\begin{itemize}
	\item Gestion des clients
	\item Gestion des projets associées aux clients
	\item Calculs des tarifs
	\item Génération de documents
	\item Recherche
\end{itemize}


Les autres membres ont vu dans ce projet l'opportunité de développer de nouvelles compétences, aussi bien sur le plan technique qu'organisationnel. D'un point de vue technique avec la programmation du C++ et l'utilisation du framework Qt. Sur le plan organisationnel, avec la mise en place de la méthode Agile Scrum. 

Ce document fait état des parties prenantes, des outils et des méthodologies de développement appliquées au projet. Il fait le point sur les résultats obtenus au niveau de la méthodologie, du logiciel et en terme de respect du plan qualité. 