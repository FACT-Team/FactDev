\documentclass[12pt,a4paper,openany]{article}
\usepackage{lmodern}
\usepackage[svgnames]{xcolor} % Required to specify font color
\input{../../LaTexTemplate/templates/couleurs.tex}

\usepackage{makeidx}
\usepackage[utf8]{inputenc} 
\usepackage{marvosym}
\usepackage[T1]{fontenc}
\usepackage[francais]{babel}
\usepackage[top=1.7cm, bottom=1.7cm, left=1.7cm, right=1.7cm]{geometry}
\usepackage{verbatim}
\usepackage[urlbordercolor={1 1 1}, linkbordercolor={1 1 1}, linkcolor=vert1, urlcolor=bleu, colorlinks=true]{hyperref}
\usepackage{tikz} %Vectoriel
\usepackage{listings}
\usepackage{fancyhdr}
\usepackage{multido}
\usepackage{amssymb}
\usepackage{float}
\usepackage[francais]{minitoc}
\usepackage[final]{pdfpages} 
\usepackage{graphicx} % Required for box manipulation
\usepackage{makeidx}

\newcommand{\titre}{FactDev :\\\vspace{10px} Création de devis et facture} 
\newcommand{\titreFooter}{\includegraphics[width=2cm]{../../../images/FACT_official.png}~~~FactDev : Création de devis et facture} 
\newcommand{\subtitle}{Compte Rendu mensuel : Mois de Mars}
\newcommand{\auteur}{Équipe FACT}
\newcommand{\semestre}{~}
\newcommand{\annee}{2015}
\newcommand{\logo}{../../LaTexTemplate/templates/ups.jpg}


\newcommand{\pole}{}
\newcommand{\sigle}{~}
\makeindex
\usepackage[totoc]{idxlayout}


\input{../../LaTexTemplate/templates/listings.tex}
\input{../../LaTexTemplate/templates/article.tex}
\input{../../LaTexTemplate/templates/remarquesExempleAttentionArticle.tex}
\input{../../LaTexTemplate/templates/polices.tex}
\input{../../LaTexTemplate/templates/affichageChapitreArticle.tex}


\newcommand*{\plogo}{\fbox{$\mathcal{PL}$}} % Generic publisher logo
%----------------------------------------------------------------------------------------
%	TITLE PAGE
%----------------------------------------------------------------------------------------

\newcommand*{\rotrt}[1]{\rotatebox{90}{#1}} % Command to rotate right 90 degrees
\newcommand*{\rotlft}[1]{\rotatebox{-90}{#1}} % Command to rotate left 90 degrees

\newcommand*{\titleBC}{\begingroup % Create the command for including the title page in the document
\newlength{\drop} % Command for generating a specific amount of whitespace
\drop=0.1\textheight % Define the command as 10% of the total text height

\vspace*{-50px}
\rule{\textwidth}{0.4pt}\par % Thick horizontal line
\begin{tabular}{p{8cm}p{5cm}p{6cm}}
	\begin{minipage}{8cm}
		Équipe FACT\\
		\textit{Conception et développement d'applications}\\~\\
		\small
%		\Mobilefone~06~84~33~52~93\\
%		\Letter~\texttt{antoine.roquemaurel@gmail.com}\\
		\Mundus~\url{http://fact-team.github.io}
	\end{minipage} &
	& 

	\begin{minipage}{5cm}
		\begin{center}
			\includegraphics[width=5cm]{logo.jpg}\\
			\tiny{Rédigé avec \LaTeX{}\\Version du \today}
		\end{center}
	\end{minipage}
\end{tabular}


\vspace{\drop} % Whitespace between the top lines and title
\centering % Center all text

\vspace{100px}
\def\CP{\textit{\Huge \titre}} % Title

\settowidth{\unitlength}{\CP} % Set the width of the curly brackets to the width of the title
{\color{LightGoldenrod}\resizebox*{\unitlength}{\baselineskip}{\rotrt{$\}$}}} \\[\baselineskip] % Print top curly bracket
\textcolor{Sienna}{\CP} \\[\baselineskip] % Print title
{\color{RosyBrown}\Large \subtitle} \\ % Tagline or further description
{\color{LightGoldenrod}\resizebox*{\unitlength}{\baselineskip}{\rotlft{$\}$}}} % Print bottom curly bracket

\vfill % Whitespace between the title and the author name


{
\normalsize \LARGE Université Toulouse III -- Paul Sabatier}\\ % Author name

\vfill % Whitespace between the author name and the publisher logo
\Large \today % Year published

\rule{\textwidth}{0.4pt}\par % Thick horizontal line

\endgroup}

%----------------------------------------------------------------------------------------
%	BLANK DOCUMENT
%----------------------------------------------------------------------------------------


\makeatother
\includeonly {
}
\begin{document}
	\thispagestyle{empty} % Removes page numbers
	\titleBC 
	\newpage
	\setcounter{tocdepth}{1}
	\setcounter{secnumdepth}{3}
	
	\tableofcontents
	\newpage
	\section{Le logiciel : FactDev}
	FactDev est un logiciel de Facture \& Devis développé par l'équipe FACT dans le cadre d'un projet de Master à l'Université Paul Sabatier composé de : 
	\begin{itemize}
		\item \textbf{F}lorent Berbie
		\item \textbf{A}ntoine de Roquemaurel
		\item \textbf{C}édric Rohaut
		\item Andriamihary Manan\textbf{T}soa Razanajatovo
	\end{itemize}

	Plus d’informations sur \Mundus~\url{http://fact-team.github.io}

	Notre enseignant tuteur est Frédéric \bsc{Migeon}.

	\begin{figure}[H]
		\centering
		\includegraphics[width=6cm]{../FACTDev.png}
		\caption{Logo de FactDev}
	\end{figure}

 	\section{Période couverte}
	Du 01 Mars 2015 au 31 mars 2015.

	\section{Résumé des travaux de la période}
	\subsection{Sprint 4}
	Le sprint 4 est le premier de la seconde Release. Il se compose d'\textit{« issues »} provenant de bugs ou remarques de monsieur Migeon lors du sprint précédent. En plus de révisions ergonomiques et de résolution de bugs, monsieur Migeon nous a demandé de réaliser un diaporama en vue de la soutenance. 
	A ceci s'ajoute de nouvelles fonctionnalités à implanter comme décrit ci-dessous. 
	\begin{description}
		\item [Base de données] Au démarrage, l'utilisateur peut choisir entre deux types de base de données (local ou centralisé). S'il choisit le type local alors la base de données est créer localement (comme c'était déjà le cas auparavant). S'il choisit centralisé alors les informations pourront être stockées sur une machine distante afin d'être accessible depuis plusieurs postes utilisateur.  
		\item [Facture payée] On peut mentionner qu'une facture est payée. Il n'est alors plus possible de la modifier.
		\item [PDF] Il est maintenant possible d'ouvrir les factures/devis créés au format PDF.
		\item [Copier un devis/facture] \`A partir d'une facture ou d'un devis déjà existant il est possible d'en créer une nouvelle semblable. Cela permet pour des projets semblables de retrouver des prestations identiques sans avoir à les re-saisir. 
		\item [Statistiques] Calcul d'informations tel que le chiffre d'affaire pour un projet ou sur l'ensemble des projets d'un client. On peut également calculer le chiffre d'affaire d'une période. 
	\end{description}
		
	\subsection{Sprint 5}
	Nous sommes actuellement à la moitié du sprint 5. Celui-ci doit :
	\begin{description}
			\item [Bugs] Correction de bugs
			\item [Client] Ajouter et modifier des informations sur les clients (complément d'adresse, logo de l'entreprise...)
			\item [Facture] Modification des factures en fonction des nouvelles informations du client
			\item [Statistiques] Fournir un ensemble de statistiques (nombre de projets par client, nombre de factures par projet...)
		\end{description}
	
	\section{Travaux effectivement réalisés en fin de période}\label{work}
	Concernant le Sprint 4, l'ensemble des « \textit{issues} » ont été réalisées dans les temps ainsi que le diaporama demandé par monsieur Migeon.
	En plus de ce que nous avions initialement prévu, nous avons modifié l'interface afin qu'elle s'adapte automatiquement en fonction de la résolution de l'écran. 
	
	Le sprint 5 n'est pour l'instant pas terminé. Nous avons corrigé une partie des bugs existants. Nous avons ajouté les informations au client et adapté celles-ci aux factures. Il reste donc à corriger quelques bugs et à faire des fonctions statistiques. 

	\section{Charge de travail pour le groupe}
	\subsection{Charge estimé}
	\begin{table}[H]
		\centering
		\begin{tabular}{l|c|c}
			\textbf{Désignation} & \textbf{Fréquence} & \textbf{Total}\\
			\hline
			Réunions tuteur & 1h / semaine & 3h\\
			Réunions de travail & 4h / semaine & 12h\\
			Travail personnel & 8h / semaine & 24h
		\end{tabular}
		\caption{Charge de travail constatée}
	\end{table}

	\subsection{Charge constatés}
	\begin{table}[H]
		\centering
		\begin{tabular}{l|c|c}
			\textbf{Désignation} & \textbf{Fréquence} & \textbf{Total}\\
			\hline
			Réunions tuteur & 1h / semaine & 3h\\
			Réunions de travail & 4h / semaine & 12h\\
			\textbf{Travail personnel} & \textbf{10h / semaine} & \textbf{30h}
		\end{tabular}
		\caption{Charge de travail estimée}
	\end{table}
	Pour ce quatrième et cinquième Sprint nous avons décidé de consacrer le même temps que celui constaté lors des premiers rendus. 
	
	La charge de travail a été sous-estimé car certaines modifications concernées des parties du logiciel avec une forte dépendance. De plus, l'ajout d'un nouveau système de gestion de base de données demandait de doubler les tests sur la base de données afin d'être sûr que cela fonctionne en SQLite et en MySQL.  

	
	\section{Problèmes techniques constatés}
	Les problèmes techniques constatés sont tous du même ordre : ils sont dus à une dépendance forte entre plusieurs composants du logiciel. Pour cela, on s'est fixé de décomposer, au fur des nouvelles issues, le code qui concerne l'issue afin qu'il y ai moins de dépendances. 
	
	Ces modifications posent également des problèmes lors des intégrations où l'on a des conflits entre les fichiers du fait que chaque membre de l'équipe est amené à travailler sur des mêmes parties du logiciel.

	\section{Décisions prises}
	Face aux problèmes rencontrés, nous avons décomposé des parties du logiciel en sous-parties. 
	\begin{itemize}
		\item Par exemple, pour l'interface principale de l'application, nous avons séparé la partie recherche (une barre de recherche avec des filtres). 
		\item De la même manière, l'utilisateur et les clients ayant des données proches, nous avons homogénéisé certaines méthodes. 
		\item Nous avons également créer plusieurs << Widgets >> graphiques avec des méthodes génériques qui sont appelé pour d'autres fenêtres.
	\end{itemize}

	
\end{document}
