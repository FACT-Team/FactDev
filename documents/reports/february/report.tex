\documentclass[12pt,a4paper,openany]{article}
\usepackage{lmodern}
\usepackage[svgnames]{xcolor} % Required to specify font color
\input{../../LaTexTemplate/templates/couleurs.tex}

\usepackage{makeidx}
\usepackage[utf8]{inputenc} 
\usepackage{marvosym}
\usepackage[T1]{fontenc}
\usepackage[francais]{babel}
\usepackage[top=1.7cm, bottom=1.7cm, left=1.7cm, right=1.7cm]{geometry}
\usepackage{verbatim}
\usepackage[urlbordercolor={1 1 1}, linkbordercolor={1 1 1}, linkcolor=vert1, urlcolor=bleu, colorlinks=true]{hyperref}
\usepackage{tikz} %Vectoriel
\usepackage{listings}
\usepackage{fancyhdr}
\usepackage{multido}
\usepackage{amssymb}
\usepackage{float}
\usepackage[francais]{minitoc}
\usepackage[final]{pdfpages} 
\usepackage{graphicx} % Required for box manipulation
\usepackage{makeidx}

\newcommand{\titre}{FactDev :\\\vspace{10px} Création de devis et facture} 
\newcommand{\titreFooter}{\includegraphics[width=2cm]{../../../images/FACT_official.png}~~~FactDev : Création de devis et facture} 
\newcommand{\subtitle}{Compte Rendu mensuel : Mois de Février}
\newcommand{\auteur}{Équipe FACT}
\newcommand{\semestre}{~}
\newcommand{\annee}{2015}
\newcommand{\logo}{../../LaTexTemplate/templates/ups.jpg}


\newcommand{\pole}{}
\newcommand{\sigle}{~}
\makeindex
\usepackage[totoc]{idxlayout}


\input{../../LaTexTemplate/templates/listings.tex}
\input{../../LaTexTemplate/templates/article.tex}
\input{../../LaTexTemplate/templates/remarquesExempleAttentionArticle.tex}
\input{../../LaTexTemplate/templates/polices.tex}
\input{../../LaTexTemplate/templates/affichageChapitreArticle.tex}


\newcommand*{\plogo}{\fbox{$\mathcal{PL}$}} % Generic publisher logo
%----------------------------------------------------------------------------------------
%	TITLE PAGE
%----------------------------------------------------------------------------------------

\newcommand*{\rotrt}[1]{\rotatebox{90}{#1}} % Command to rotate right 90 degrees
\newcommand*{\rotlft}[1]{\rotatebox{-90}{#1}} % Command to rotate left 90 degrees

\newcommand*{\titleBC}{\begingroup % Create the command for including the title page in the document
\newlength{\drop} % Command for generating a specific amount of whitespace
\drop=0.1\textheight % Define the command as 10% of the total text height

\vspace*{-50px}
\rule{\textwidth}{0.4pt}\par % Thick horizontal line
\begin{tabular}{p{8cm}p{5cm}p{6cm}}
	\begin{minipage}{8cm}
		Équipe FACT\\
		\textit{Conception et développement d'applications}\\~\\
		\small
%		\Mobilefone~06~84~33~52~93\\
%		\Letter~\texttt{antoine.roquemaurel@gmail.com}\\
		\Mundus~\url{http://fact-team.github.io}
	\end{minipage} &
	& 

	\begin{minipage}{5cm}
		\begin{center}
			\includegraphics[width=5cm]{logo.jpg}\\
			\tiny{Rédigé avec \LaTeX{}\\Version du \today}
		\end{center}
	\end{minipage}
\end{tabular}


\vspace{\drop} % Whitespace between the top lines and title
\centering % Center all text

\vspace{100px}
\def\CP{\textit{\Huge \titre}} % Title

\settowidth{\unitlength}{\CP} % Set the width of the curly brackets to the width of the title
{\color{LightGoldenrod}\resizebox*{\unitlength}{\baselineskip}{\rotrt{$\}$}}} \\[\baselineskip] % Print top curly bracket
\textcolor{Sienna}{\CP} \\[\baselineskip] % Print title
{\color{RosyBrown}\Large \subtitle} \\ % Tagline or further description
{\color{LightGoldenrod}\resizebox*{\unitlength}{\baselineskip}{\rotlft{$\}$}}} % Print bottom curly bracket

\vfill % Whitespace between the title and the author name


{
\normalsize \LARGE Université Toulouse III -- Paul Sabatier}\\ % Author name

\vfill % Whitespace between the author name and the publisher logo
\Large \today % Year published

\rule{\textwidth}{0.4pt}\par % Thick horizontal line

\endgroup}

%----------------------------------------------------------------------------------------
%	BLANK DOCUMENT
%----------------------------------------------------------------------------------------


\makeatother
\includeonly {
}
\begin{document}
	\thispagestyle{empty} % Removes page numbers
	\titleBC 
	\newpage
	\setcounter{tocdepth}{1}
	\setcounter{secnumdepth}{3}
	
	\tableofcontents
	\newpage
	\section{Le logiciel : FactDev}
	FactDev est un logiciel de Facture \& Devis développé par l'équipe FACT dans le cadre d'un projet de Master à l'Université Paul Sabatier composé de : 
	\begin{itemize}
		\item \textbf{F}lorent Berbie
		\item \textbf{A}ntoine de Roquemaurel
		\item \textbf{C}édric Rohaut
		\item Andriamihary Manan\textbf{T}soa Razanajatovo
	\end{itemize}

	Plus d’informations sur \Mundus~\url{http://fact-team.github.io}

	Notre enseignant tuteur est Frédéric \bsc{Migeon}.

	\begin{figure}[H]
		\centering
		\includegraphics[width=6cm]{../FACTDev.png}
		\caption{Logo de FactDev}
	\end{figure}

 	\section{Période couverte}
	Du 29 Janvier 2014 au 28 Février 2015.

	\section{Résumé des travaux de la période}
	\subsection{Sprint 2}
	Le sprint 2 est composé des \textit{« issues »} préalablement définies et des nouvelles résultant de la première revue de sprint. Ces nouvelles \textit{« issues »} correspondent à des bugs ou des modifications ergonomiques au logiciel. 
	Au niveau du sprint 1 nous pouvions créer un nouveau Client et le modifier. Dans le sprint 2, il doit être possible d'ajouter un nouveau projet à un client particulier ou, si le projet existe, de le modifier. A chaque projet d'un client est associé une facture et/ou un devis. Ceci engendre des modifications au niveau de l'interface graphique :
	\begin{itemize}
		\item Vue de la liste des clients (existante depuis le sprint 1)
		\item Vue de la liste des projets pour un client
		\item Vue de la liste des factures et/ou devis pour un projet d'un client
		\item Vue hiérarchique avec l'ensemble des clients, ses projets et factures/devis associés
	\end{itemize}
	Afin de s'assurer que l'utilisateur saisi des données correctes on lui empêche de sauvegarder si certains champs sont vide ou erronées. Les champs de saisie possèdent une icône indiquant la validité ou non des informations qu'ils contiennent. 
	
	\subsection{Sprint 3}
	Le sprint 3 est le dernier de notre première « Release ». La fonctionnalité majeure de ce sprint consiste à créer une nouvelle/facture ou devis avec, pour chacune, les informations suivantes:

	\begin{itemize}
		\item un \textbf{titre}
		\item une \textbf{date}
		\item une \textbf{description}
		\item la \textbf{liste des projets} concernés avec, pour chacun, un tarif (journalier/horaire)
		\item la \textbf{liste des prestations} pour chaque projet avec le nombre d'heures/jours consacrés à la réalisation de chaque prestation
	\end{itemize}
	
	Une fois les informations saisies il est possible de générer la facture au format pdf.
	Dans le cas d'un devis, on doit également pouvoir le modifier ce qui n'est pas le cas d'une facture accessible uniquement en lecture.
	La barre de recherche s'est également améliorée afin de tenir compte des nouvelles modifications, à savoir la possibilité de rechercher par : 
	\begin{itemize}
		\item \textbf{Société} Nom de la société
		\item \textbf{Client} Nom du client
		\item \textbf{Projet} Nom du projet
		\item \textbf{Prestation} Description de la prestation
		\item \textbf{Facture/Devis} Titre ou numéro de la facture ou du devis
	\end{itemize}  
	
	\section{Travaux effectivement réalisés en fin de période}\label{work}
	L'ensemble des « \textit{issues} » ont été réalisées dans les temps. 

	En plus de ce que nous avions initialement prévu, nous avons ajouté l'outil d'intégration continue Travis. Il s'agit d'un outil en ligne qui compile le projet à chaque fois que celui-ci est modifié, s'assure que le code compile, que les tests passent et calcule la couverture de code. De plus, il génère la documentation associé à notre projet (Doxygen, Manuel d'utilisateur).
	A chaque nouveau \textit{ « commit »}, Travis s'exécute en fond et indique directement sur \textit{GitHub} si les tests passent et si la couverture de code a diminué ou non. Lorsqu'un utilisateur crée une \textit{« Pull Request »} il lui est impossible de fusionner celle-ci si les tests ne passent pas. 

	\section{Charge de travail pour le groupe}
	\subsection{Charge estimé}
	\begin{table}[H]
		\centering
		\begin{tabular}{l|c|c}
			\textbf{Désignation} & \textbf{Fréquence} & \textbf{Total}\\
			\hline
			Réunions tuteur & 1h / semaine & 4h\\
			Réunions de travail & 4h / semaine & 10h\\
			Travail personnel & 8h / semaine & 24h
		\end{tabular}
		\caption{Charge de travail constatée}
	\end{table}

	\subsection{Charge constatés}
	\begin{table}[H]
		\centering
		\begin{tabular}{l|c|c}
			\textbf{Désignation} & \textbf{Fréquence} & \textbf{Total}\\
			\hline
			Réunions tuteur & 1h / semaine & 4h\\
			Réunions de travail & 4h / semaine & 16h\\
			Travail personnel & 8h / semaine & 24h
		\end{tabular}
		\caption{Charge de travail estimée}
	\end{table}
	Pour ce second et troisième Sprint nous avons décidé de consacrer le même temps que celui constaté lors du premier compte rendu. 
	
	La partie consacrée à l'entraide entre les développeurs a diminué par rapport au mois précédent car les membres maîtrise davantage le langage et les outils. En revanche un temps plus important fut consacré à des choix de conception ou d'ergonomie qui justifie un temps de travail identique à celui du mois précédent.

	\section{Problèmes techniques constatés}
	Nous avons eu quelques soucis que nous avons pu résoudre.

	\subsection{Conception de l'interface}
	Lors du sprint 2, nous avions mal défini l'interface de création de Devis et Factures, ainsi lors de l'ajout des Tarifs pour un devis ou une
	facture ,nous avons été obligé de refaire toute l'interface des prestations, ce qui nous a fait perdre du temps et prendre du retard sur le
	sprint.

	\subsection{Fuites de mémoires}
	Le logiciel comportait un certain nombre de fuites de mémoire qui rendait impossible l'utilisation de la fonction recherche. Afin de régler ce
	problème, nous avons utilisé l'outil Valgrind, qui détecte les endroits où une variable n'est pas désalloué, à l'aide de celui-ci, les fuites ont pu
	être comblées.

	\subsection{Déploiement sous Mac}
	Un certain nombre de problèmes sont apparus pour le déploiement sous Mac OS, principalement du à des erreurs de bibliothèques.

	\section{Décisions prises}
	\subsection{Intégration continue}
	Afin d'avoir un projet qui soit le plus stable possible, et de limiter au maximum les problèmes d'intégration, nous avons choisi d'utiliser
	Travis pour faire de l'intégration continue comme détaillé section \ref{work}.

	Ainsi, lors de la revue de code, notre \textit{« Pull Request »} contient les indicatifs de Travis, une fonctionnalité ne peut être intégrée que si tous les
	indices sont au vert, soit : 
	\begin{itemize}
		\item Le code compile correctement
		\item Les tests passent
		\item La couverture de code n'a pas diminué par rapport à la version précédente
		\item La documentation est correctement générée
	\end{itemize}

	\subsection{Couverture des tests}
	Afin que notre code soit le mieux testé possible, nous avons pris deux décisions : 
	\begin{itemize}
		\item Ne pas tester l'interface graphique, en raison du temps et de la difficulté que cela nous prendrait.
		\item Assurer une couverture de code d'au moins 80\%
	\end{itemize}
\end{document}
