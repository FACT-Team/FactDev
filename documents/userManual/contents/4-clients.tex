\chapter{Les clients}
Les clients sont le cœur de l'application, elle a donc besoin d'ajout de nouveau clients pour pouvoir être utilisée. 
\begin{figure}[H]
	\centering
	\includegraphics[width=10cm]{screens/clients.png}
	\caption{Gestion des clients}
\end{figure}

\section{Liste des clients}
La liste des, cf. figure 3.1, est accessible dès l’ouverture du logiciel. Celle-ci se trouve au centre du logiciel. 

La liste des clients contient uniquement les informations permettant de facilement les identifier à savoir, le nom de la société, le nom,
prénom, le numéro de téléphone et l’adresse email. La sélection dans le tableau de l’un des clients permet, via le panneau du client,
d’obtenir les informations détaillées sur celui-ci. 

\section{Ajout d'un client}
L'ajout d’un patient peut se faire soit via le menu <<Client $\rightarrow$ Nouveau Client>>, via la barre d'outils ou encore via le bouton situé sous la
liste. 
\begin{figure}[H]
	\centering
	\includegraphics[width=10cm]{screens/ajouterClient.png}
	\caption{Ajouter un client}
\end{figure}

\section{Édition d'un client}
L’édition d’un client a pour but de corriger d’eventuels erreurs sur les informations d’un client. Pour ce faire, il suffit de sélectionner
un client dans le tableau et de cliquer sur le bouton modifier situé sous la liste des clients ou, via le menu contextuel, lors d’un clic
droit sur le client dans le tableau. 
 La fenêtre est similaire à celle lors d’un ajout de client, seul diffère les champs qui sont déjà pré-remplis. 

