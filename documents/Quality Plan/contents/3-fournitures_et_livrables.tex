\chapter{Fournitures et Livrables\index{Livrables}}
Durant le projet, l’équipe fournira plusieurs livrables pour l’ensemble des parties prenantes et des enseignants de l’UE Projet :

\begin{itemize}
	\item Plan d'assurance Qualité Logicielle (signé par le client et déposé sur Moodle)\index{Livrables!Plan d'assurance qualité}
	\item Compte-rendu mensuel d'activité (envoyé par courriel au responsable de l'UE)\index{Livrables!Compte rendu menseul}
	\item Bilan de projet\index{Livrables!Bilan du projet}
	\item Graphiques d’avancement\index{Livrables!Graphiques d'avancement}
	\begin{itemize}
		\item Burndown chart\index{Livrables!Graphiques d'avancement!Burdown chart}
		\item Burnup chart\index{Livrables!Graphiques d'avancement!Burnup chart}
	\end{itemize}
	\item Manuel de l’utilisateur\index{Livrables!Manuel de l'utilisateur}
	\item Carnet des produits \key{Product Backlog}\index{Livrables!Product Backlog}
	\item Revues de sprint\index{Méthode Scrum!Revue de sprint}
	\item Builds\footnote{Un \key{build} est un \key{artefact} logiciel autonome résultant de conversion de fichiers de code source en code exécutable}\index{Livrables!Build}
\end{itemize}

Le résultat du projet sera présenté par le Titulaire au Client lors des versions livrables. La première version livrable du logiciel se fera le 26 février 2015 et la seconde le 24 avril 2014. La validité du logiciel se fera à l'appréciation de notre client, Monsieur \bsc{Migeon}.

L'organisation, le déroulement du projet et les résultats obtenus feront l'objet d'un oral lors de la soutenance. 