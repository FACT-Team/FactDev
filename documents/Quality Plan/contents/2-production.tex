\chapter{Production}
\section{Méthode de développement et espace de travail}
\subsection{Méthode Scrum}
Le développement du projet se fera selon la méthode Agile << Scrum >>, comme cela a été convenu par notre encadrant.

Cetté méthode, basée sur les stratégies itératives et incrémentales, permet de produire à la fin de chaque << sprint >> (incrément/itération) une version testable du logiciel. Les différents événements associés à Scrum accroît la communication grâce à des réunions quotidiennes aussi appelées << mélées >>. Ceci une meilleur cohésion, une meilleure coopération et une meilleure homogénéité du travail fournit par les membres de l'équipe. A cela s'ajoute la présence d'<<artefacts>>, c'est-à-dire des éléments à réaliser avec des ordres de priorité et qui contribuent à améliorer la productivité. 
                    
Dans le cadre de ce projet, la méthode Scrum s’avère être tout à fait pertinent et ceux pour plusieurs raisons.  Première, dans la mesure où nous avons proposé un sujet et spécifié les fonctionnalités de celui-ci, la méthode Scrum se veut adapté. En effet, les fonctionnalités que nous avons proposé permettent de définir les limites de notre premier version << Release >> livrable. L’ajout de fonctionnalité en fonction des attentes du client pourront être implémentés au fur et à mesure des différents << sprints >>.  Cela a pour avantage de fournir un travail continu, d’assurer un suivi avec le client pour répondre au mieux à ses besoins. De plus, la durée du projet étant relativement courte, il serait difficile de revenir sur notre conception préalable alors qu’ici chaque << sprint >> permet de s’assurer que le projet avance dans la bonne direction.  

Outre les avantages qu’apporte la méthode Scrum à ce projet, l’équipe avait la volonté d’évoluer vers une méthode qui diffère de celles qui ont pu être aborder en cours. Cette volonté est d’ailleurs conforter par le désire de découvrir  de nouvelles technologies (C++ et Qt jusque-là peu connus par la majorité du groupe) et de nouvelles méthodologies. 

\subsection{Organisation et rôles dans l'équipe de développement}

Contrairement aux méthodes classiques, la méthode Scrum définit des rôles pour chacun des membres de l'équipe. Les rôles des membres de notre équipe ont été définit comme suit: 

\textbf{Scrum Master : Florent Berbie } \\  
La mission principale du Scrum Master est de s'assurer que les membres de l'équipe respecte les conventions imposées par la méthode Scrum. C'est à lui de veiller aux respect des conventions par les membres. Il se doit de jouer un rôle de meneur lors de phases importantes telles que le planning poker ou encore les mêlées quotidiennes.
        
\textbf{Product owner : Antoine de Roquemaurel} \\
Le << Product owner >> est la seule personne responsable du carnet de produit et de sa gestion. Ce carnet comprend l'expression de tous les éléments avec le niveau de priorité qui leurs sont associés. Cela compte également leur difficulté de réalisation ainsi que son importance vis-à-vis du client). La bonne compréhension des éléments ainsi que leurs réalisation est de la responsabilité du << Product owner >>.
                    
\textbf{Équipe de développement: Florent Berbie, Antoine de Roquemaurel, Cédric Rohaut, Andriamihary Razanajatovo} \\
Au sein de l'équipe de développement des rôles spécifiques ont été attribués à chacun des membres :
                    
\textbf{Chef de projet : Nom Prénom} \\                   
% TODO
%Le chef de projet assurera le bon déroulement du projet dont il a la charge. Vis-à-vis du client, il est responsable de l’intégralité du projet comprenant les travaux réalisés, les délais, les résultats et les fournitures destinées au client. Vis-à-vis du responsable de l’UE Projet, il est le garant du bon fonctionnement de l’équipe et devra l’avertir de tout problème majeur dans le déroulement du projet (abandon d’un membre de l’équipe, client indisponible, par exemple).
                    
\textbf{Directeur qualité : Cédric Rohaut} \\
Le responsable qualité sera garant de la transposition des exigences du client sous forme de solutions technique au sein du logiciel. En d'autres termes, il veillera à ce que le logiciel apporte une solution technique optimale pour le client. 

\textbf{Directeur communication : Nom Prénom} \\
% TODO
%Le responsable communication sera chargé d’envoyer, de recevoir ainsi que de relayer les messages provenant des différentes parties prenantes et notamment entre le client et l’équipe de développement. Il représentera et parlera au nom de l’équipe Sodium dans les e-mails ou les échanges téléphoniques par exemple.        
                    
\textbf{Directeur documentation : Andriamihary Razanajatovo} \\
Le responsable documentation sera chargé de la révision de l'ensemble des documents (documentation du logiciel y compris) avant leur remise ou leur soumisson aux parties prenantes concernées. Il veillera à la qualité des fondamentaux, à savoir le fond et la forme, des documentations à fournir. 

\textbf{Directeur technique Qt, C++ : Nom Prénom (A supprimer ?)} \\
%Le responsable technique Android-Java sert de support à l’équipe en cas de problèmes techniques liés au développement sur la plate-forme Android. De par ses connaissances acquises dans ce domaine, il sera le plus à même à aider les membres de l’équipe projet ayant des difficultés avec cette technologie.

%De plus, dans le but de minimiser les couplages lors de la conception, il veillera à ce que le patron MVC soit correctement utilisé. Ainsi, le code sera clairement découpé en trois parties : modèle, vue et contrôleur ce qui permettra de s’assurer des contrôles d'interactions entre nos composants logiciels.
                    
%Avoir deux responsables techniques différents couvrant à eux deux les technologies de l’application Android et celle de l’interface web nous assure de ne pas nous retrouver dans des situations critiques de problèmes techniques. Cela nous permettra donc un avancement constant du projet.
                    
%Dans un souci d’assurance qualité, chaque responsable sera en mesure de présenter les différentes techniques qui leur étaient possibles d’utilisation lors des différentes phases, et justifiera le choix de la technique adoptée en présentant lors des préparations des stories pour le backlog de sprint, les raisons et les critères de détermination.
                    
%L’architecture technique du futur produit sera documentée et réalisée par les deux responsables techniques pour leur partie respective. D’un point de vue des tests, la responsabilité des différents tests à mener sera rattachée au responsable technique web pour la partie “site” et au responsable technique Android-Java pour la partie “application”.

\subsection{Outils pour la gestion de projet Agile}
Faire un résumé avec GitHub et les réunions avec Migeon
GitHub permet d’avoir accés en direct au projet. Grâce aux “issues”, il est possible de suivre les tâches qui sont en cours de résolution ou celles qui ont déjà été résolues. Ainsi, si le client souhaite s’informer sur l’avancement du logiciel, il peut y accéder facilement via GitHub.

