\chapter{Planning}
Conformément aux dates qui nous sont imposées, le projet a débuté le 27 décembre 2014 et prendra fin juste après la soutenance d'avril.

La première phase du projet consiste à faire d'abord le choix des langages et de l'environnement de développement. Une fois défini, on réalise la conception globale de l'application. De plus, afin que les membres de l'équipe aillent dans le même sens et aient une même vision de ce à quoi doit ressembler le logiciel, on réalise des maquettes de l'interface du logiciel. 

Conformément à la méthode \key{Scrum}\index{Méthode Scrum}, des réunions hebdomadaires avec le client ont été programmées. De plus, l'équipe de développement organisera des \key{mêlées}\index{Méthode Scrum!Mêlées} quotidiennes afin de faire un point sur l'avancement du \key{Sprint}\index{Méthode Scrum!Sprint} en cours. 

Le Titulaire prendra compte des remarques et demandes de notre encadrant Monsieur \bsc{Migeon} lors des réunions. Celles-ci seront l'objet d'une ou plusieurs \key{issue}\index{Méthode Scrum!Issue} du \key{Sprint} suivant. 

Le projet comportera deux \key{Releases}\index{Méthode Scrum!Release} avec, pour d'elle, trois \key{Sprints} durant chacun deux semaines comme le montre le diagramme de GANTT ci-dessous.