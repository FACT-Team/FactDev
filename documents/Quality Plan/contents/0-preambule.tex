\chapter*{Avant-Propos}

Le Plan d'assurance Qualité Logicielle a pour objectif la définition et le suivi des dispositions à prendre dans le cadre du projet \FactDev{} afin d’en
assurer la qualité, une bonne gestion et d’atteindre les résultats attendus.

À cet effet, le Plan d’Assurance Qualité Logicielle fixe les droits, devoirs et responsabilités de chaque partie prenante en vue d’assurer l'atteinte
de ces objectifs.

Il constitue un outil de travail et un référentiel commun à tous les acteurs pour leur donner une vision similaire du projet, mais il constitue
également le cahier des charges de la qualité, réalisé en collaboration avec le client puis approuvé par celui-ci. Il constitue enfin la définition
des procédures à suivre, des outils à utiliser, des normes à respecter, de la méthodologie de développement du produit et des contrôles prévus pour
chaque activité.

Ainsi, d’un commun accord sont déterminés ces différents aspects du projet qui constituent un contrat entre le titulaire et le client et toutes les
autres parties prenantes. Ce contrat prend effet dès son acceptation par le client et les personnes concernées et peut être, si les circonstances
l'obligent, amené à être modifié au cours du projet. Dans ce cas, toute évolution future sera soumise à l'acceptation du client car au terme du
projet, le Plan d’Assurance Qualité Logicielle constituera l’un des documents de résultat du projet.
