\chapter*{Avant-Propos}

\setlength{\parindent}{1cm}
Dans le cadre d'un projet, le Plan d'assurance Qualité Logicielle détermine les mesures nécessaires permettant de répondre aux exigences de qualité et de gestion inhérentes au projet. 
Ainsi, le Plan d’Assurance Qualité Logicielle définit les droits,les devoirs et les responsabilités de chaque partie prenante afin d’assurer le respect de ces exigences. 

Il constitue un outil de travail et un référentiel commun à tous les acteurs pour leur donner une vision collective du projet. \\ 
Il est également le cahier des charges de la qualité et est réalisé en collaboration avec le client puis approuvé par celui-ci. \\
Enfin, il définit les procédures à suivre, les outils à utiliser, les normes à respecter, la méthodologie de développement du produit et les contrôles prévus pour
chaque activité.

Le Plan d'Assurance qualité constitue un contrat entre le titulaire, le client et toutes les autres parties prenantes. 
Ce contrat prend effet dès son acceptation par le client et les personnes concernées et peut être, si les circonstances l'exigent, amené à être modifié au cours du projet. Dans ce cas, toute évolution future sera soumise à l'acceptation du client. En effet, au terme du projet, le Plan d’Assurance Qualité Logicielle constituera l’un des documents de résultat du projet.
